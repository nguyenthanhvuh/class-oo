\documentclass[11pt]{article}
\usepackage[margin=1in]{geometry}
\usepackage{graphicx}
\usepackage{booktabs}
\usepackage[colorlinks,pagebackref,pdfusetitle,urlcolor=blue,citecolor=blue,linkcolor=red,bookmarksnumbered,plainpages=false]{hyperref}

\begin{document}
\begin{center}
{\Large \textsc{SWE-419 (Fall '24): OO Software Design and Implementation}}
\end{center}
\begin{center}
Course Page \url{https://nguyenthanhvuh.github.io/class-oo}
\end{center}



\begin{minipage}[t]{.75\textwidth}
    \centering
\begin{tabular}{llcccll}
  \toprule
  \textbf{Meetings:} & Fri 10:30 PM -- 1:10 PM  & & & & \textbf{Place:} & Horizon Hall 2010 and \href{https://piazza.com/gmu/fall2024/swe419nguyen}{Piazza}\\
\textbf{Instructor:} & \href{https://go.gmu.edu/tvn}{ThanhVu Nguyen} & & &  & \textbf{Email:} &  \href{mailto:tvn@gmu.edu}{tvn@gmu.edu}\\
\textbf{Office Hr:} & Mon 2:00PM -- 3:00PM & & & & \textbf{Place:} & ENGR 4430\\
                     &(email to confirm)&&&&&\\
  \hline
\textbf{GTA:} & Hamaad Ullah Zuberi & & &  & \textbf{Email:} &  \href{mailto:hzuberi@gmu.edu}{hzuberi@gmu.edu}\\
\textbf{Office Hr:} & TBD & & & & \textbf{Place:} & TBD \\
  \hline
\textbf{UTA:} & Josabeth Zelalem Hailu & & &  & \textbf{Email:} &  \href{mailto:jhailu2@gmu.edu}{jhailu2@gmu.edu}\\
\bottomrule
\end{tabular}
\end{minipage}


\section{Description}

   To give the students a solid understanding of modern software construction. To encourage the construction of software systems of high quality. In-depth study of software construction in a modern language including program specifications and OO designs. Concepts such as information hiding, data abstraction, and object-based and object-oriented software construction are discussed and illustrated. This course is part of the core of the SWE program.

\subsection{Prerequisite}
\begin{itemize}
\item SWE Foundation Courses or equivalent.
\item OOP Language (e.g., Python or Java)
\end{itemize}


\subsection{Course Materials}
\begin{itemize}
  \item Barbara Liskov with John Guttag. \href{https://www.amazon.com/Program-Development-Java-Specification-Object-Oriented/dp/0201657686/ref=sr_1_2?dchild=1&qid=1626231221&refinements=p_27\%3ABarbara+Liskov&s=books&sr=1-2&text=Barbara+Liskov}{\emph{Program
  Development in Java: Abstraction, Specification, and Object-Oriented  Design}}. Addison Wesley, 2001, ISBN 0-201-65768-6.  (\textbf{Required}, free through \href{https://learning-oreilly-com.mutex.gmu.edu/library/view/program-development-in/9780768685299/ch1.html}{ O'Reilly}


\item Joshua Bloch.
\href{https://www.amazon.com/Effective-Java-Joshua-Bloch/dp/0134685997/ref=sr_1_1?dchild=1&keywords=effective+java&qid=1626231154&sr=8-1}{\emph{Effective
  Java}}. Third Edition. Addison-Wesley Professional, 2017, ISBN
  978-0-13-468599-1. (\textbf{Recommended}, free through
        \href{https://learning-oreilly-com.mutex.gmu.edu/library/view/effective-java-3rd/9780134686097/cover.xhtml}{O'Reilly})

\end{itemize}

\paragraph{Note:} Don't worry about the publication date of Liskov, which is basically a math book and therefore ages well.

\section{Weekly Schedule}



This class is a group-based, \emph{in-person} offering. You need to be
present in class at class time. You will also need to schedule regular
meetings with your group.

Each week will cover a topic, which is a small number of related
technical issues (see course \href{./schedule.html}{schedule}). Each
topic will follow roughly the same sequence of preparation, in-class
learning activities, homework completion and (possibly) presentation,
and knowledge assessment. Many of these activities will be group based.
The sequence is:

\begin{itemize}
\item
  Preparation: Complete assigned readings and watch relevant videos, if
  any. No formal submission for this activity.
\item
  Class Meeting:

  \begin{itemize}
  \item
    Combined session on the day\textquotesingle s topic, followed by
    in-class exercises (group breakouts)
  \item
    Break
  \item
    Combined session on the day\textquotesingle s topic, may followed by
    in-class exercises (group breakouts)
  \end{itemize}
\item
  Homework (group-based)
\item
  Assessments via quiz: takes place at the \textbf{end} of class the
  same day as the homework is due.
\end{itemize}

\subsection{Grading}\label{grading}

\begin{tabular}{lr}
\textbf{Assignments} & \textbf{Percentage} \\
\hline
Group Functioning (group-based) & 5\% \\
Homework assignments (group-based) & 35\% \\
Weekly Quizzes (individual) & 35\% \\
  Final exam (individual) & 25\% \\
  \hline
    \textbf{Total} & 100\% \\
\end{tabular}

\subsubsection{Scale}\label{scale}

\begin{tabular}{llllll}
A+ & $\ge$ 97\% & A & $\ge$  93\% & A- & $\ge$
90\% \\
B+ & $\ge$  87\% & B & $\ge$  83\% & B- & $\ge$
80\% \\
C & $\ge$  70\% & D & $\ge$  60\% & F & $<$
60\% \\
\end{tabular}

\subsubsection{Group Functioning}\label{group-functioning}

Every student needs to be part of a group. I would prefer that groups
stay stable throughout the semester, but if there is a good rationale to
reconfigure a group, we'll do that.



\paragraph{Creating groups} You will have a chance to form your own group.
If you can't find one, we can help. Each has should have \textbf{2 to 4 students}. If your group dwindles to just yourself, you'll need to join another group.

At the end of the semester, each individual will provide an assessment
of the rest of their group. This assessment will determine the ``Group Functioning'' part of the grade.

\subsubsection{Homework Assignments}\label{homework-assignments}

There are weekly \emph{group homework assignments}, which are given through the class \href{./schedule.html}{schedule web site}. Your group will submit assignments online.

Because of the way in which this class is taught, it is important to
stay on pace. Homework assignments are due \textbf{before class}. Late submissions
are not accepted except in truly exceptional circumstances.

Some important notes:

\begin{itemize}
\item
Each group should be prepared to present their homework solution in
class.

\item
  \textbf{Statement of who did what}. Homeworks are group exercises.
  Each submission must contain a specific statement of who did what.
\item
  There are \textbf{no make-ups}.
\item
  Other than the first assignment (where we might not have formed all
  groups), only one submission per \textbf{\textbf{group}}. Everyone in
  the group gets the same credit.
\end{itemize}

\subsubsection{Weekly Quizzes}\label{weekly-quizzes}

We will have a quiz every week. The quiz will be based on the material covered in the previous weeks.
Each quiz happens during the last 15--20 minutes of class.


\paragraph{Grading Make-up Policy} You will have the opportunity to make up a quiz if you miss it or do poorly. The grading and make-up policy is as follows:

\begin{itemize}
\item
  All quizzes count towards the final grade. Each quiz is scored on a 10
  point scale. Missed quizzes score 0/10. Students who miss a quiz or
  perform badly on a quiz may choose to take the make-up.
\item
  The maximum possible score on the make-up is 8/10. (Example: your quiz
  grade is 5/10. You take the make-up and correctly answer 9 of 10
  equally weighted questions. Your final score improves from 5/10 to
  8/10.)
\item
  If you attempt the make-up, that score counts, no matter what your
  score was on the quiz. (Example: your quiz grade is 7/10. You take the
  make-up and correctly answer 5 of 10 equally weighted questions. Your
  final score declines from 7/10 to 5/10.)
\item
  Scheduling: the GTA will offer the make-up during TA office hours. The
  make-up can be different than the quiz given in class, but focuses on
  the same topics.
\item
  The make-up must be taken promptly and within a window of two class
  meetings from the quiz. (Example: Quiz 1 takes place on Wednesday,
  September 1. The make-up must be taken on or before Wednesday,
  September 15. Another Example: Quiz 11 takes place on Wednesday,
  November 11. Because we don't meet the week of
  Thanksgiving, the make-up must be taken on or before Wednesday,
  December 1.)
\item
  Each quiz only has one make-up, and you can only attempt that make-up
  once. However, you are free to use the make-up mechanism on as many
  different quizzes as you wish.
\item
  Quizzes are generally returned one week after the quiz is taken.
  Make-ups are returned after the window has closed.
\end{itemize}

\subsubsection{Final Exam}\label{final-exam}

There will be an final exam at the time specified by the
university's final exam schedule.

\begin{center}\rule{0.5\linewidth}{0.5pt}\end{center}

\subsection{Class Attendance}\label{class-attendance}

I place great emphasis on peer learning and interactive engagement. The
class is structured to leverage group interactions to the largest extent
possible for the purpose of maximizing learning gain through out the
semester.

\textbf{Bottome line}: It's important to be in class.

\subsection{In-Class Exercises}\label{in-class-exercises}

I plan an in-class exercise for every class. Students will work in their
designated group. Very often, the in-class exercises will be closely
related to an upcoming homework assignment.

\subsection{Record Keeping}\label{record-keeping}

We'll use Blackboard (or Canvas) to maintain scores and
attendance data. Grades are computed according to this syllabus.
It's the student's responsibility to
ensure that your grade records are correct.

\begin{center}\rule{0.5\linewidth}{0.5pt}\end{center}


% \section{Links}
% \label{sec:org63a886d}
% \begin{itemize}
% \item \href{assignments.org}{Schedule and Assignments}
% \item \href{project.org}{Project Info}
% \end{itemize}

% \subsection{Related courses}
% \begin{itemize}
% \item \href{https://www.sri.inf.ethz.ch/teaching/reliableai21}{Eth Zurich Reliable and Trustworthy Artificial  Intelligence}
% \end{itemize}

\section{GMU Policies}
\subsection{Honor Code}\label{sec:honor-code}

As with all GMU courses, this class governed by the \href{https://academicstandards.gmu.edu}{GMU Honor Code}. In this course, all assignments carry with them an implicit statement that it is the sole work of the author.

\subsection{Learning Disabilities}

Disability Services at George Mason University is committed to providing equitable access to learning opportunities for all students by upholding the laws that ensure equal treatment of people with disabilities. If you are seeking accommodations for this class, please first visit \url{https://ds.gmu.edu/} for detailed information about the Disability Services registration process. Then please discuss your approved accommodations with me. Disability Services is located in Student Union Building I (SUB I), Suite 2500. Email: \href{mailto:ods@gmu.edu}{ods@gmu.edu} | Phone: (703) 993-2474


\end{document} 
