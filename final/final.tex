\documentclass[10pt]{article}
\usepackage[utf8]{inputenc}
\usepackage[T1]{fontenc}
\usepackage[top=1in,bottom=1in,left=1in,right=1in]{geometry}
\usepackage{amsmath,amssymb}
\usepackage{booktabs}


\usepackage{xcolor}
\definecolor{dkgreen}{rgb}{0,0.5,0}
\definecolor{dkred}{rgb}{0.5,0,0}
\definecolor{dkgray}{rgb}{0.3,0.3,0.3}

\usepackage{listings}
\lstset{basicstyle=\sffamily\small,
  language=C,
  keywordstyle=\color{blue},
  ndkeywordstyle=\color{red},
  commentstyle=\color{dkgreen},
  stringstyle=\color{dkred},
  numbers=left,
  numberstyle=\ttfamily\footnotesize\color{dkgray},
  stepnumber=1,
  numbersep=8pt,
  backgroundcolor=\color{white},
  tabsize=2,
  showspaces=false,
  showstringspaces=false,
  emph={try,catch}, emphstyle=\color{red}\bfseries,
  breaklines,
  breakatwhitespace,
  mathescape,
  literate={"}{{\ttfamily"}}1
  {<-}{$\leftarrow$}2
  {!=}{$\neq$}1,
  columns=flexible,
  morekeywords={then,end,do},
}

\usepackage{etoolbox}
\newtoggle{usecoe}
\settoggle{usecoe}{false} % not CoE version
\newcommand{\coe}[1]{\iftoggle{usecoe}{#1}{}}
\newcommand{\notcoe}[1]{\nottoggle{usecoe}{#1}{}}
\newcommand{\coeite}[2]{\iftoggle{usecoe}{#1}{#2}}
\newcommand{\classname}{SWE619--Fall'21}

\author{Your Name:}
\date{}
\title{\classname{}: Final Exam}

\begin{document}
\maketitle
\section*{Instructions}

\begin{enumerate}
\item This is an open-book/open-notes exam. This means that you can access course materials on paper or on the internet..

\item It is a violation of the honor code to communicate with any other person (except me, the instructor) about this exam while you are taking it.

\item It is a violation of the honor code to discuss or share the contents of this exam in any way with any student who is currently registered for this course but who has not yet completed this exam.

%When you see a reference to code, you should map that to the relevant Java file I provided as part of your study guide for this exam. 

\item Capture your answers as a PDF document and submit on Blackboard by the deadline. If, for any reason, you have a problem submitting to BB,  submit your final on Piazza in a private post.  Your post should also explain your problem.
\end{enumerate}

\begin{center}
  \begin{tabular}{lcc}
    \toprule
    Section		&	Points	&	Score \\
    \midrule
    Question 1	&	 30	& \\
    Question 2	&	 35	& \\
    Question 3 	&	 35	& \\
    \midrule
    Total		&	100	& \\
    \bottomrule
  \end{tabular}
\end{center}

\newpage
\section{Question 1}
Consider {\tt Queue.java}.

\begin{enumerate}
\item
Write a reasonable \texttt{toString()} implementation.

\item
Consider a new method, \texttt{deQueueAll()}, which does exactly  what the name suggests. Write a reasonable contract for this method and then implement it.  Be sure to follow Bloch's advice with respect to generics.

\item
Rewrite the \texttt{deQueue()} method for an immutable version of this class.

\item
Write a reasonable implementation of \texttt{clone()}. 


\end{enumerate}
\newpage
\section{Question 2}

Consider Bloch's final version of his {\tt Chooser} example, namely {\tt GenericChooser.java}.


\begin{enumerate}
\item
What would be a good representation invariant for this class?
You may want to argue that the client
view of {\tt Chooser} objects  should be changed to support 
your proposed invariant.
If so, explain exactly what you are doing and why.
\item
Supply suitable contracts for the constructor and the {\tt choose()} method
and recode if necessary.
The contracts should be consistent with your answer to the previous question.
Explain exactly what you are doing and why.
\item
Argue that the {\tt choose()} method, as documented and possibly updated
in your previous answers, is correct.  
You don't have to be especially formal, but you do have
to ask (and answer) the right questions.  
\end{enumerate}

\newpage
\section{Question 3}

Consider {\tt StackInClass.java}.
You should take special note of the {\tt push()} method;
it is a variation on Bloch's code.

\begin{enumerate}
\item
What is wrong with {\tt toString()}?  Fix it.

\item
As written, {\tt pushAll()} requires documentation that violates encapsulation.  Explain why
and then write a contract for {\tt pushAll()}.

\item
Rewrite the {\tt pop()} method for an immutable version of the {\tt Stack} class.
Keep the same instance variables.

\item
Implementing the {\tt equals()} method for this class is a messy exercise, but would
	be much easier if the array was replaced by a list.  Explain why.
	Note:  You are not required to provide a implementation in your answer,
	but if you find it helpful to do so, that's fine.
\end{enumerate}

\newpage
\section{Question 4}

Consider the program below. 
\begin{lstlisting}
{N >= 0}   // precondition
i := 0 ;

while(i < N)
  i := N;

{i == N} // post condition
\end{lstlisting}


\begin{enumerate}
\item Informally argue that this program is correct with respect to the given specification (pre/post conditions).
\item Compute several loop invariants for this program (e.g., 3 should suffice). For each loop invariant, informally argue why it is a loop invariant.  
\item \emph{Sufficiently strong loop invariants:}  Use a sufficiently strong loop invariant to formally prove that the program is correct with respect to given specification. This loop invariant can be one of those you computed in the previous question or something new.
  \begin{itemize}
    \item }Note: show all works for this step (e.g., obtain weakest preconditions, verification condition, and analyze the verification condition).
  \item Recall that if the loop invariant is strong enough, then you will be able to do the proof. In contrast, if it is not strong enough, then you cannot do the proof.
  \end{itemize}
\item \emph{Insufficiently strong loop invariants:} Use another loop invariant (could be one of those you computed previously) and show that you cannot use it to prove the program.  Also, show all work as the previous question.
\end{enumerate}


\newpage
\section{Question 5}

This question helps me determine the grade for group functioning.  It does not affect the grade of this final.

\begin{enumerate}
\item Who are your group members?
\item For each group member, rate their participation in the group on the following scale:
  \begin{enumerate}
  \item Completely absent
  \item Occasionally attended, but didn't contribute reliably
  \item Regular participant; contributed reliably
  \end{enumerate}
\end{enumerate}
\end{document}
