\documentclass[10pt]{article}
\usepackage[utf8]{inputenc}
\usepackage[T1]{fontenc}
\usepackage[top=1in,bottom=1in,left=1in,right=1in]{geometry}
\usepackage{amsmath,amssymb}
\usepackage{booktabs}
\usepackage{hyperref}

\usepackage{xcolor}
\definecolor{dkgreen}{rgb}{0,0.5,0}
\definecolor{dkred}{rgb}{0.5,0,0}
\definecolor{dkgray}{rgb}{0.3,0.3,0.3}

\usepackage{listings}
\lstset{basicstyle=\sffamily\small,
  language=C,
  keywordstyle=\color{blue},
  ndkeywordstyle=\color{red},
  commentstyle=\color{dkgreen},
  stringstyle=\color{dkred},
  numbers=left,
  numberstyle=\ttfamily\footnotesize\color{dkgray},
  stepnumber=1,
  numbersep=8pt,
  backgroundcolor=\color{white},
  tabsize=2,
  showspaces=false,
  showstringspaces=false,
  emph={try,catch}, emphstyle=\color{red}\bfseries,
  breaklines,
  breakatwhitespace,
  mathescape,
  literate={"}{{\ttfamily"}}1
  {<-}{$\leftarrow$}2
  {!=}{$\neq$}1,
  columns=flexible,
  morekeywords={then,end,do},
}

\usepackage{etoolbox}
\newtoggle{usecoe}
\settoggle{usecoe}{false} % not CoE version
\newcommand{\coe}[1]{\iftoggle{usecoe}{#1}{}}
\newcommand{\notcoe}[1]{\nottoggle{usecoe}{#1}{}}
\newcommand{\coeite}[2]{\iftoggle{usecoe}{#1}{#2}}
\newcommand{\classname}{SWE619--Fall'21}

\author{Your Name:}
\date{}
\title{\classname{}: Final Exam}

\begin{document}
\maketitle
\section*{Instructions}

\begin{enumerate}
\item This is an open-book/open-notes exam. This means that you can access course materials on paper or on the internet..

\item It is a violation of the honor code to communicate with any other person (except me, the instructor) about this exam while you are taking it.

\item It is a violation of the honor code to discuss or share the contents of this exam in any way with any student who is currently registered for this course but who has not yet completed this exam.

%When you see a reference to code, you should map that to the relevant Java file I provided as part of your study guide for this exam. 

\item Capture your answers as a PDF document and submit on Blackboard by the deadline. If, for any reason, you have a problem submitting to BB,  submit your final on Piazza in a private post.  Your post should also explain your problem.

\end{enumerate}


\begin{center}
  \begin{tabular}{lcc}
    \toprule
    Section		&	Points	&	Score \\
    \midrule
    Question 1	&	 25	& \\
    Question 2	&	 25	& \\
    Question 3 	&	 25	& \\
    Question 4 	&	 25	& \\
    Question 5 	&	 0	& \\
    Question 6 	&	 0	& \\
    \midrule
    Total		&	100	& \\
    \bottomrule
  \end{tabular}
\end{center}

\newpage
\section{Question 1}
Consider \href{https://nguyenthanhvuh.github.io/class-oo/files/Queue.java}{Queue.java}.



\begin{enumerate}
\item
Write a reasonable \texttt{toString()} implementation. Explain what you did

\item
Consider a new method, \texttt{deQueueAll()}, which does exactly  what the name suggests. Write a reasonable contract for this method and then implement it.  Be sure to follow Bloch's advice with respect to generics. 
Explain what you did

\item
Rewrite the \texttt{deQueue()} method for an immutable version of this class. Explain what you did

\item
Write a reasonable implementation of \texttt{clone()}. Explain what you did. 


\end{enumerate}
\newpage
\section{Question 2}

Consider Bloch's final version of his Chooser example, namely  \href{https://nguyenthanhvuh.github.io/class-oo/files/GenericChooser.java}{GenericChooser.java}.


\begin{enumerate}
\item
What would be a good representation invariant for this class?
You may want to argue that the client
view of {\tt Chooser} objects  should be changed to support 
your proposed invariant.
If so, explain exactly what you are doing and why.
\item
Supply suitable contracts for the constructor and the {\tt choose()} method
and recode if necessary.
The contracts should be consistent with your answer to the previous question.
Explain exactly what you are doing and why.
\item
Argue that the {\tt choose()} method, as documented and possibly updated
in your previous answers, is correct.  
You don't have to be especially formal, but you do have
to ask (and answer) the right questions.  
\end{enumerate}

\newpage
\section{Question 3}

Consider \href{https://nguyenthanhvuh.github.io/class-oo/files/StackInClass.java}{StackInClass.java}.
Note of the {\tt push()} method is a variation on Bloch's code.

\begin{enumerate}
\item
What is wrong with {\tt toString()}?  Fix it.

\item
As written, {\tt pushAll()} requires documentation that violates encapsulation.  Explain why
and then write a contract for {\tt pushAll()}.

\item
Rewrite the {\tt pop()} method for an immutable version of the {\tt Stack} class.
Keep the same instance variables. Rewrite what you did.

\item
Implementing the {\tt equals()} method for this class is a messy exercise, but would
	be much easier if the array was replaced by a list.  Explain why.
	Note:  You are not required to provide a implementation in your answer,
	but if you find it helpful to do so, that's fine.
\end{enumerate}

\newpage
\section{Question 4}

Consider the program below (\texttt{y} is the input). 
\begin{lstlisting}
{y $\ge$ 1}   // precondition

x := 0;
while(x < y)
  x += 2;

{x $\ge$ y} // post condition
\end{lstlisting}


\begin{enumerate}
\item Informally argue that this program satisfies the given specification (pre/post conditions).
\item Give 3 loop invariants for this program. For each loop invariant, informally argue why it is a loop invariant.  
\item \emph{Sufficiently strong loop invariants:}  Use a sufficiently strong loop invariant to formally prove that the program is correct with respect to given specification. This loop invariant can be one of those you computed in the previous question or something new.
  \begin{itemize}
    \item Note: show all works for this step (e.g., obtain weakest preconditions, verification condition, and analyze the verification condition).
  \item Recall that if the loop invariant is strong enough, then you will be able to do the proof. In contrast, if it is not strong enough, then you cannot do the proof.
  \end{itemize}
\item \emph{Insufficiently strong loop invariants:} Use another loop invariant (could be one of those you computed previously) and show that you cannot use it to prove the program. 
  \begin{itemize}
    \item Note: show all work as the previous question.
  \end{itemize}
\end{enumerate}


\newpage
\section{Question 5}

This question helps me determine the grade for group functioning.  It does not affect the grade of this final.

\begin{enumerate}
\item Who are your group members?
\item For each group member, rate their participation in the group on the following scale:
  \begin{enumerate}
  \item Completely absent
  \item Occasionally attended, but didn't contribute reliably
  \item Regular participant; contributed reliably
  \end{enumerate}
\end{enumerate}



\newpage
\section{Question 6}


There is no right or wrong answer for the below questions, but they can help me improve the class. 
I might present your text verbatim (but anonymously) to next year's students when they are considering taking the course (e.g., in the first week of class) and also add your advice to the project description pages.

\begin{enumerate}
\item What were your favorite \textbf{and} least aspects of this class? Favorite topics?
\item Favorite things the professor did or didn't do?
\item What would you change for next time?
\end{enumerate}

\end{document}
